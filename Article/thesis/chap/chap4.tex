\chapter{总结与展望}

本文首先从两个方向上给出了定位论文图片中公式位置的方法,其一是使用目标
检测算法从段落图片中定位公式,其二是先进行单词分割,再将单词图片利用CNN分
类,最后再利用单词图片位置信息还原。本文简述了目标检测的经典算法,然后详
细实现了第二种方法,并对该方法进行了优化,使得结果很好。实现的具体步骤首先
是进行数据的准备,我们的图片都是从latex源码经处理后编译为pdf再转化为png,生
成的每张图片都有两个版本,白色框版本用来作为数据,红色框版本用来生成数据标
注。在利用网络进行训练之前首先对论文图像进行预处理,即单词分割,包括文行分
割和文行内单词分割两个步骤,都是利用空白来进行分割然后进行筛选。文行筛选最
重要的指标是文行的前四分之一平均行和,辅助标准为平均行和和高度。单词分割则
是主要利用空白的宽度经过最小二乘法得到单词间空白,注意要舍取最大的两个空白
来得到更佳的效果。在参考了CPTN的方法后,将长单词图片分割为方形单词图片来
保留更多的单词特征。单词分割完后就要生成数据库,由于单词图片和非单词图片之
间的数量差距比较大,需要先进行采样处理,分别尝试了过采样和欠采样。采样完后
生成tfrecords文件作为网络的输入。网络结构一开始采用了LeNet的7层结构,之后参
考AlexNet和VGGNet进行了改进,结合自己的实际情况采用了10层的网络结构,卷积
核都使用了小尺寸。网络中使用了多种优化算法,PReLu、指数衰减学习率、滑动平
均、正则化、dropout、批标准化等。一共使用了三种方法生成数据来进行训练,一是
过采样,一是欠采样,而是欠采样基础上使用方形单词分割。一开始使用了错误的标
签来进行训练,结果过采样的效果非常差,但欠采样和方形单词分割仍能得到还不错
的结果。使用正确的标签来进行训练后,三个方法的效果都变得非常好,数据结果上
来看方形单词分割优于只欠采样优于过采样。虽然测试集的单词图片数有一两万张,
但实际的论文图片只有47张,故通过实际比对各方法在实际论文图片上的效果,并和
原本的带红框的原始标注进行对比,找出了错误的地方并进行了相应的分析。这个方
法相比使用目标检测有许多显著的优点,一是精确度比较高,绝大多数检测对象都能
够准确识别类别,且位置信息非常精确,因为是由实际单词分割给出位置而不是网络
学习找出位置。另外网络结构简单,训练所需时间非常短。

尽管本文的在测试集上的效果已经非常显著,但仍有一些固有问题。如文行分
割和单词分割不够精确,始终会有漏查单词或者检测多余的行间公式。另外有的单
个单词图片难以辨别是否是公式,尤其是字母图片,缺乏上下文信息来进一步判断。
若要进一步工作,试图使用更好的方法来区别文行和行间公式,如继续使用神经网
络来学习文行和非文行的特征。要更精确地判断单词图片是否为公式,则可以尝试
如CTPN中使用LSTM等RNN来获得序列信息,从而提高判断能力。