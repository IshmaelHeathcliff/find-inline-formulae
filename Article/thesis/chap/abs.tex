

\begin{cabstract}
  公式是我们在文本中比较关心的对象,而行间公式,即单独处于一行中的公式在很显著的位置,行内公式,即处于文本段落中的公式则不容易快速分辨。现在的在线科技资料大都以PDF或图片的形式存在,那么问题就变成了从文本图片中定位行内公式。若把这看作一个目标检测的问题,则可以使用经典的目标检测算法去解决这个问题,从整张文本图片或者段落图片中直接框定行内公式的位置,这么做的好处在于将这个问题泛化为端对端的问题,可以直接使用现有模型和方法去完成,减少繁杂的预处理。但文本公式检测问题有其特殊性,如文本图片的格式是规整的,不像自然场景图片那么复杂,如果能够针对文本图片公式检测这个问题的特点来制定解决办法,那么可以期望得到比直接使用目标检测算法更好的结果。为此我们先做图像预处理,将图片进行单词分割,得到单个单词图片,再使用分类器CNN对单词图片进行分类,选出公式图片。由于只进行图片分类,简单的网络结构就能获得不错的结果,大大减少了网络训练的时间,从而可以使用更多的数据来达到更好更广泛的效果。但同时,数据预处理就变得复杂,而且难以有通用的方法,针对各种特殊情况都要有相应的解决方法。故本文的主要工作为优化数据处理,然后采用CNN网络进行分类,最后得到了不错的实际效果。这个方法也只针对文本图片有效,无法很好地推广到其他问题上去。
\end{cabstract}

\begin{eabstract}
The formula is the object we care about in the text. While the inter-row formula is in a very prominent position, the in-line formula is not easy to distinguish quickly. Most of the current online scientific and technical information exists in the form of PDFs or images, so the problem becomes the positioning of the in-line formula from text Images. If think of this as a problem of object detection, we can use the classic object detection algorithms to solve this problem and directly frame the position of the formulae from the whole image or paragraph images. The advantage of this method is to generalize the problem to an end-to-end problem, which can be directly completed by using existing models and methods to reduce complicated preprocessing. However, formula detection has its particularity. For example, the format of the text image is regular. It is not as complicated as the natural scene image. By formulating a solution according to the characteristics of formula detection, we can expect better results than using the existing object detection algorithms. To do this, we first do image preprocessing, divide the image into word images, and then use the classifier CNN to classify the word images and select out the formula images. Since we only need images classification, a simple network structure can get a good result as well. And the time spent on network training is greatly reduced, so that more data can be used to achieve better and general results. At the same time, however, data preprocessing becomes complicated, and it is difficult to have a common method, and there must be a corresponding solution for various special cases. So the main work of this thesis is to optimize the methods of data preprocessing, and use a proper CNN to classify the images, in the end we get a not bad result in reality. This method is valid only for the image of
text, and can not be well promoted to other issues.
\end{eabstract}