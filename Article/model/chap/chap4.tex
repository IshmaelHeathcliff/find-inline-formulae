

\chapter{结果和展望}

本文首先从两个方向上给出了定位论文图片中公式位置的方法,其一是使用目标检测算法从段落图片中定位公式,其二是先进行单词分割,再将单词图片利用CNN分类,最后再利用单词图片位置信息还原。本文简述了目标检测的经典算法,然后详细实现了第二种方法,并对该方法进行了优化,使得结果很好。实现的具体步骤首先是进行数据的准备,我们的图片都是从latex源码经处理后编译为pdf再转化为png,生成的每张图片都有两个版本,白色框版本用来作为数据,红色框版本用来生成数据标注。在利用网络进行训练之前首先对论文图像进行预处理,即单词分割,包括文行分割和文行内单词分割两个步骤,都是利用空白来进行分割然后进行筛选。文行筛选最重要的指标是文行的前四分之一平均行和,辅助标准为平均行和和高度。单词分割则是主要利用空白的宽度经过最小二乘法得到单词间空白,注意要舍取最大的两个空白来得到更佳的效果。在参考了CPTN的方法后,将长单词图片分割为方形单词图片来保留更多的单词特征。单词分割完后就要生成数据库,由于单词图片和非单词图片之间的数量差距比较大,需要先进行采样处理,分别尝试了过采样和欠采样,结果后者的效果好很多。采样完后生成tfrecords文件作为网络的输入。网络结构一开始采用了LeNet的7层结构,之后参考AlexNet和VGGNet进行了改进,结合自己的实际情况采用了10层的网络结构,卷积核都使用了小尺寸。网络中使用了多种优化算法,PReLu、指数衰减学习率、滑动平均、正则化、dropout、批标准化等。



