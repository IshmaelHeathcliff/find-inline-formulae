% Copyright (c) 2014,2016 Casper Ti. Vector
% Public domain.

\begin{cabstract}
	公式是我们在论文中比较关心的对象,而行间公式在很显著的位置,行内公式则不容易快速分辨。现在的在线论文资料大都以pdf或图片的形式存在,那么问题就变成了从论文的图片中定位行内公式。若把这看作一个目标检测的问题,则可以使用经典的目标检测算法去解决这个问题,从整张或者段落图片中直接框定行内公式的位置,这么做的好处在于将这个问题泛化为端对端的问题,可以直接使用现有理论和方法去完成,减少繁杂的预处理。但这个问题有其特殊性,如论文图片的格式是规整的,不像自然场景图片那么复杂,如果能够针对论文图片和这个问题的特点来制定解决办法,那么可以期望得到比直接使用目标检测算法更好的结果。为此我们先做图像预处理,将图片进行单词分割,再使用分类器CNN等对单词图片进行分类,选出公式图片,这样网络部分将变得简单,使用简单的网络结构就能获得不错的结果,大大减少了网络训练的时间,从而可以使用更多的数据来达到更好更广泛的效果。但同时,数据预处理就变得复杂,而且难以有通用的方法,针对各种特殊情况都要有相应的解决方法。故本文的主要工作为优化数据处理,然后采用针对性的CNN网络进行分类,最后得到了不错的实际效果。这个方法也只针对论文图片有效,无法很好地推广到其他问题上去。但针对这个问题则预期有更好的效果。
\end{cabstract}

\begin{eabstract}
	The formula is the object we care about in the paper. While the inter-row formula is in a very prominent position, the in-line formula is not easy to distinguish quickly. Most of the current online papers exist in the form of pdfs or pictures, so the problem becomes the positioning of the in-line formula from pictures of papers. If you think of this as a problem of object detection, you can use the classic object detection algorithms to solve this problem. Directly frame the position of the formulae from the whole or paragraph images. The advantage of doing this is to generalize the problem to an end-to-end problem, which can be directly completed by using existing theories and methods to reduce complicated preprocessing. However, this problem has its particularity. For example, the format of the paper is regular. It is not as complicated as the natural scene image. By formulating a solution according to the characteristics of this problem, you can expect better results than using the object detection algorithms. To do this, we first do image preprocessing, divide the image into word images, and then use the classifier CNN to classify the word images and select the formula pictures, so that the network part will become simple, and a simple network structure can get a good result as well. And the time spent on network training is greatly reduced, so that more data can be used to achieve better and general results. At the same time, however, data preprocessing becomes complicated, and it is difficult to have a common method, and there must be a corresponding solution for various special cases. So the main work of this thesis is optimize the methods of data preprocessing, and use a proper CNN to classify the pictures, in the end we get a not bad result in reality. This method is also valid only for the picture of theses, and can not be well promoted to other issues. But for this problem it is expected to have a better effect.
\end{eabstract}
